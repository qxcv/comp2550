\documentclass[10pt,a4paper]{article}

% For ex. 5
% \documentclass[10pt,twocolumn,a4paper]{article}
% \documentclass[10pt,a4paper]{book}

% For ex. 2
\usepackage{amssymb,amsmath}

% For ex. 4
% \usepackage[numbers]{natbib}
% Alternative:
\usepackage[authoryear]{natbib}

\begin{document}
\title{My First \LaTeX~Document}
\author{Sam Toyer}
\maketitle

\section*{Exercise 1}
    This is my first \LaTeX~document. Here are some forbidden characters:

    \begin{center}
        \#, \$, \%, \^{}, \&, \_, \{, \}, \~{}, \textbackslash
    \end{center}

    Now for some styles: this text is \textbf{bold}, this text is
    \textsl{slanted}, this text is \textit{italicised}, this text is
    \texttt{teletype}, and this text is \emph{emphasized}.
    \texttt{\textbackslash textit} will \textit{always make things
    \textit{italic}}, so nesting has no additional effect.
    \texttt{\textbackslash emph}, on the other hand, \emph{will make specific
    pieces of text \emph{stand out} from the surrounding text}, \emph{even
    \emph{when \emph{nested.}}}

\section*{Exercise 2}
    The quadratic equation $ax^2 + bx + c = 0$ has solution

    \[
        x = \frac{\-b \pm \sqrt{b^2 - 4ac}}{2a}
    \]

    % First example
    Let $x \in \mathbb{R}^n$ be $n$-dimensional vector. We write $x_i$ for the
    $i$-th element of $x$. The Euclidean distance between two $n$-dimensional
    vectors is

    \begin{align}
        \|x - y\| &= \sqrt{ \sum_{i=1}^{n} (x_i - y_i)^2 }
    \end{align}

    % Second example
    Let $\lambda \in \mathbb{R}$ be a scalar. We define the absolute value of
    $\lambda$ as

    \begin{align}
        \left| \lambda \right| &=
        % Neat, I didn't know about {cases}! Here I was writing out
        % \left\{\begin{array}... like a chump.
        \begin{cases}
            \lambda & \text{if $\lambda \geq 0$} \\
            -\lambda & \text{otherwise}.
        \end{cases}
    \end{align}

    % Third example
    The determinant of the $2 \times 2$ matrix

    \begin{align}
    M = \left[ \begin{matrix}
    a & b \\
    c & d
    \end{matrix} \right]
    \end{align}

    is $|M| = ad - bc$.

    % Our code
    \[
        f(x) = \frac1{2\pi} \exp\left\{-\frac{(x-\mu)^2}{2\sigma^2}\right\}
    \]

\section*{Exercise 3}
    \setcounter{section}{3}
    \subsection{The Quadratic Equation}
    \label{sec:quadratic}

    The quadratic equation is $ax^2 + bx + c = 0$.

    \subsection{Solution}
    \label{sec:solution}

    The solution of the quadratic equation defined in
    Section~\ref{sec:quadratic} is

    \begin{align}
        x &= \frac{-b \pm \sqrt{b^2 - 4ac}}{2a}
        \label{eqn:solution}
    \end{align}

    The term under the square root in Equation~\ref{eqn:solution} is known as
    the discriminant.

\section*{Exercise 4}

    Compare

    \begin{quote}
        The standard reference for \LaTeX~is \citet{Lamport:1994}. Many
        scientific articles are written using \LaTeX~\citep{Lamport:1994}.
    \end{quote}

    with

    \begin{quote}
        The standard reference for \LaTeX~is \citep{Lamport:1994}. Many
        scientific articles are written using \LaTeX~\citep{Lamport:1994}.
    \end{quote}

    I'm citing \citet{brubaker2013lost} and \citet{velaga2012improving}.
    \texttt{authoryear} has made \texttt{\textbackslash citet} include both the
    author and the year of publication for these papers.

\section*{Exercise 5}

    The \texttt{twocolumn} option gives us a nice two-column layout. The
    \texttt{book} class makes a number of changes, including giving the title
    and bibliography pages of their own and shifting other pages to the right or
    left depending on page number parity.

\bibliographystyle{abbrvnat}
\bibliography{tutorial}

\end{document}
